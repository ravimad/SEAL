\documentclass[12pt]{article}

\usepackage{amsmath}
\usepackage{amsfonts}

\begin{document}

\title{Implementation of the Composition Operation}
\maketitle

\input{genericmacros}
\newcommand{\Null}{\ensuremath{\mathit{null}}}
\newcommand{\WSR}{\textsc{WSR}}
\newcommand{\procP}{P}
\newcommand{\procQ}{Q}
\newcommand{\proc}[1]{P_{#1}}

%Latex symbols
\newcommand{\larrow}[1]{\stackrel{#1}{\rightarrow}}
\newcommand{\llangle}{\langle \! \langle}
\newcommand{\rrangle}{\rangle \! \rangle}
\newcommand{\corrabs}{\sim}

%CFG entitites
\newcommand{\entryOf}[1]{\mathit{entry}({#1})}
\newcommand{\exitOf}[1]{\mathit{exit}({#1})}
\newcommand{\exitS}{\mathit{exit}}
\newcommand{\varOf}[1]{\varphi_{#1}}
\newcommand{\avarOf}[1]{\vartheta_{#1}}
\newcommand{\stmtedge}[3]{{#1} \larrow{#2} {#3}}
\newcommand{\edge}[2]{{#1} \rightarrow {#2}}

%syntactic entities
\newcommand{\Vars}{\syn{Vars}}
\newcommand{\Globals}{\syn{Globals}}
\newcommand{\Locals}{\syn{Locals}}
\newcommand{\Params}{\syn{Params}}
\newcommand{\Fields}{\syn{Fields}}
\newcommand{\Labels}{\syn{Labels}}
\newcommand{\newst}{\syn{NewStmt}}
\newcommand{\ldst}{\syn{LoadStmt}}
\newcommand{\paramOf}[1]{\syn{Param}(#1)}

%Auxiliary functions
\newcommand{\alloc}{\mathit{Alloc^{\sharp}}}
\newcommand{\load}{\mathit{Load^{\sharp}}}
\newcommand{\initbind}{\mathit{InitBind^{\sharp}}}
\newcommand{\lift}[1]{\hat{#1}}

%semantic fucntions
%\newcommand{\concsem}[1]{\sembrack{#1}_c}
\newcommand{\concsem}[1]{\sembrack{#1}^{\natural}}
\newcommand{\abssem}[1]{\sembrack{#1}^{\sharp}}
\newcommand{\gabssem}[1]{\sembrack{#1}_a}
\newcommand{\simplify}{\mathit{Simplify}}
\newcommand{\nmOf}[1]{\mathit{NM_{#1}}}
\newcommand{\mnodesOf}[1]{\mathit{NodesToMerge_{#1}}}
\newcommand{\garcol}{\mathit{GC^{\natural}}}
\newcommand{\absGarcol}{\mathit{GC^{\sharp}}}
\newcommand{\push}[1]{\mathit{push}_{#1}}
% \newcommand{\pop}[2]{\mathit{pop}_{#1}^{#2}}
\newcommand{\pop}[1]{\mathit{pop}_{#1}}
\newcommand{\callret}[1]{\mathit{Call}_{#1}}
\newcommand{\abspush}[1]{\mathit{apush}_{#1}}
\newcommand{\abspop}[2]{\mathit{apop}_{#1}^{#2}}
\newcommand{\translate}[1]{\mathit{trans_{#1}}}
\newcommand{\removeEscNodes}{\mathit{removeNonEscaping}}

%xforms
\newcommand{\xform}[2]{{#1}\langle{#2}\rangle}
\newcommand{\xformop}{\langle \rangle}
%\newcommand{\absxform}[3]{{#1}\llangle{#2}\rrangle^{#3}_a}
\newcommand{\absxform}[3]{
\ifthenelse{\equal{#3}{}} 
    {{#1}\llangle{#2}\rrangle_{a}} 
    {{#1}\llangle{#2}\rrangle_{#3}}
}
\newcommand{\absxformop}{\llangle \rrangle}
\newcommand{\pxform}[3]{{#1} \llangle {#2,#3} \rrangle}
\newcommand{\mabsxform}[2]{{#1}\llangle{#2}\rrangle}
\newcommand{\paramxform}[3]{\overline{{#1}\llangle{#2}\rrangle}_{#3}}
\newcommand{\append}{\mathit{Append}}
%\newcommand{\abxform}[2]{{#1}\llangle{#2}\rrangle}
%\newcommand{\concxform}[3]{{#1}\llangle{#2}\rrangle^{#3}_c}
%\newcommand{\concxformop}{\llangle \rrangle_c}

% CFD: Concrete Functional Domain
\newcommand{\CFD}{{\mathcal F}_c}
\newcommand{\leCFD}{\sqsubseteq_c}
\newcommand{\leCFDT}{\leCFD^*}
\newcommand{\joinCFD}{\sqcup_c}
\newcommand{\bigjoinCFD}{{\textstyle \bigsqcup_c}}

% AFD: Abstract Functional Domain
\newcommand{\AFD}{{\mathcal F}_a}
\newcommand{\leCO}{\sqsubseteq_{co}}
\newcommand{\leCOT}{\leCO^*}
\newcommand{\joinCO}{\sqcup_{co}}
\newcommand{\bigjoinCO}{{\textstyle \bigsqcup_{co}}}

% CGD: Concrete Graph Domain
\newcommand{\CGD}{\mathbb{G}_c}
% AGD: Abstract Graph Domain
\newcommand{\AGD}{\mathbb{G}_a}

% \G: a single graph
\newcommand{\G}{g}
% \GS: a set of graphs
\newcommand{\GS}{G}
% \T: a transformer graph
\newcommand{\T}{\tau}
% \CG: a concrete graph
\newcommand{\CG}{ptg}

% \AllV: universal set of vertices
\newcommand{\AllV}{N_c}
\newcommand{\AbsAllV}{N_a}
% \V: vertex set
\newcommand{\V}{\mathsf{V}}
\newcommand{\Vof}[1]{\V_{#1}}
\newcommand{\node}[1]{n_{#1}}
% \E: edge set
\newcommand{\E}{\mathsf{E}}
% IV, IE, EV, EE: internal/external vertex/edge sets
% \newcommand{\IV}{\textsc{IV}}
% \newcommand{\IE}{\textsc{IE}}
% \newcommand{\EV}{\textsc{EV}}
% \newcommand{\EE}{\textsc{EE}}
\newcommand{\IV}{\mathsf{IV}}
\newcommand{\IE}{\mathsf{IE}}
\newcommand{\EV}{\mathsf{EV}}
\newcommand{\EE}{\mathsf{EE}}
\newcommand{\paramState}{\pi}
\newcommand{\HB}{\leadsto}
\newcommand{\tedge}[3]{\langle #1,#2,#3 \rangle}

%identity transformer
\newcommand{\TID}{\mathit{\T_{id}}}

\newcommand{\Ttuple}{(\EV,\EE,\paramState,\IV,\IE,\sigma,\HB)}
\newcommand{\TtupleSub}[1]{(\EV_{#1},\EE_{#1},\paramState_{#1}, \IV_{#1},\IE_{#1}, \sigma_{#1}, \HB_{#1})}
\newcommand{\GtupleSub}[1]{(\V_{#1},\E_{#1},\sigma_{#1})}

\newcommand{\inEscapeSet}{\mathit{Escaping}}
%\newcommand{\outEscapeSet}{\mathit{OutEscaping}}
%\newcommand{\escapeSet}{\mathit{Escaping}}
%\newcommand{\escapes}[1]{\mathit{escapes}_{#1}}

%mapping operators
\newcommand{\fmap}{\eta}
\newcommand{\fmapOf}[2]{\eta\sembrack{#1,#2}}

%partial orders
\newcommand{\embeds}{\preceq}
\newcommand{\partialembeds}[1]{\preceq^{#1}}

%concretization function
\newcommand{\gammaG}{\gamma_G}
\newcommand{\gammaT}{\gamma_T}

% AFDP: Abstract Funtional Domain with partitions
\newcommand{\equivOf}[1]{\simeq_{#1}}
\newcommand{\AFDP}{{\mathcal F}_m}
\newcommand{\AFDPT}{\AFDP^*}
\newcommand{\PartFunc}{\mathcal{P}}
\newcommand{\PF}{\xi}
\newcommand{\lePart}{\le_{p}}
\newcommand{\joinPart}{\sqcup_{p}}
\newcommand{\leAFDP}{\le_{m}}
\newcommand{\leAFDPT}{\le_{m}^*}
\newcommand{\joinAFDP}{\sqcup_{m}}
\newcommand{\gammaM}{\gamma_M}
\newcommand{\pabssem}[1]{\sembrack{#1}_m}
\newcommand{\pabsxform}[3]{{#1}\llangle{#2}\rrangle^{#3}_m}

\newcommand{\absCall}[3]{\mathit{Call}_{#3}^{\sharp}({#1},{#2})}
\newcommand{\absPush}[1]{\mathit{push}_{#1}^{\sharp}}
\newcommand{\absPop}[1]{\mathit{pop}_{#1}^{\sharp}}
\newcommand{\absfmap}{\fmap_{a}}
\newcommand{\absfmapOf}[2]{\fmap_{a}\sembrack{#1,#2}}
\newcommand{\includesOf}[1]{\includes\sembrack{#1}}
\newcommand{\includes}{\mathit{Incl}}

%specialized semantics
\newcommand{\absspecial}{\absdomainOf{\Omega}}
\newcommand{\abssemspecial}[1]{\abssemOf{\Omega}{#1}}
\newcommand{\absCallSpecial}[3]{\absCallOf{\Omega}{#1}{#2}{#3}}
\newcommand{\fmapspecial}{\fmap_{\Omega}}

%parametrized EV creation
\newcommand{\paramtrans}[1]{\mathit{\overline{trans}_{#1}}}
\newcommand{\transload}{\syn{Load^{\flat}}}
\newcommand{\mulift}{\mathit{\mu_{\uparrow}}}
\newcommand{\paramafmap}{\overline{\fmap}_{a}}
\newcommand{\paramsimp}{\overline{\simplify}}
\newcommand{\paramincl}{\overline{\includes}}
\newcommand{\paramfmapsp}{\overline{\fmap}_{\Omega}}

%parametrized domain elements
\newcommand{\absdomain}{\mathcal{A}}
\newcommand{\absdomainOf}[1]{\mathcal{A}_{#1}}
\newcommand{\abssemOf}[2]{\sembrack{#2}_{#1}}
\newcommand{\abstg}[1]{\mathit{\rho}_{#1}}
\newcommand{\absID}{\mathsf{ID}^{\sharp}}
\newcommand{\IDof}[1]{\mathsf{ID}_{#1}}
\newcommand{\absCallOf}[4]{\mathit{Call^{#1}_{#4}}(#2,#3)}
\newcommand{\absgamma}[1]{\mathit{\gamma_{#1}}}
\newcommand{\absalpha}[1]{\mathit{\alpha_{#1}}}


%WSR analysis
\newcommand{\WSRtuple}{(\EV,\EE, \IV,\IE, \sigma)}
\newcommand{\WSRtupleSub}[1]{(\EV_{#1},\EE_{#1}, \IV_{#1},\IE_{#1}, \sigma_{#1})}

%Buss et al analysis
\newcommand{\buss}{\textit{Buss \etal{}}}
\newcommand{\approxOf}[1]{\Delta_{S}}

%Lattner et al analysis
\newcommand{\lattner}{\textit{Lattner \etal{}}}
\newcommand{\gammabar}{\overline{\gamma}}
\newcommand{\alphabar}{\overline{\alpha}}

%Cheng et al analysis
\newcommand{\cheng}{\textit{Cheng \etal{}}}
\newcommand{\AP}{\mathit{AP}}
\newcommand{\RAP}{\mathit{RAP}}
\newcommand{\LAP}{\mathit{LAP}}
\newcommand{\rot}{\mathit{Root}}
\newcommand{\APL}{\mathit{AP_{L}}}
\newcommand{\RAPL}{\mathit{RAP_{L}}}
\newcommand{\LAPL}{\mathit{LAP_{L}}}
\newcommand{\ap}{\delta}
\newcommand{\aptwo}{\theta}
\newcommand{\bound}{\mathit{bound}}
\newcommand{\lconcat}{\mathit{lconcat}}
\newcommand{\rconcat}{\mathit{rconcat}}
\newcommand{\sptr}{\mathit{S_{PTR}}}
\newcommand{\prefix}{\mathit{Prefix}}
\newcommand{\allaps}{\mathit{AllAPs}}
\newcommand{\eval}{\mathit{Eval}}
\newcommand{\replace}[1]{\mathit{replace}_{#1}}
%\newcommand{\nodes}{\mathit{Nodes}}
%\newcommand{\simplifyOf}[1]{\mathit{Simplify_{#1}}}
%\newcommand{\EAP}{\mathit{EAP}}

Consider the $\simplify$ operation defined as follows:
Let $\T = (\EV, \EE,\paramState,\IV,IE,\sigma)$ and $\V = \EV \cup \IV$.
%
\begin{align}
x \in \V & \Rightarrow  x \in \mu(x) \\
\nonumber
\langle u,f,v \rangle \in \IE, \langle u',f,v' \rangle \in \EE, \\
\mu(u) \cap \mu(u') \ne \emptyset 
& \Rightarrow \mu(v) \subseteq \mu(v') \label{eq:alias-rule}
\end{align}
%
Let $\fmap$ denote the lfp value of $\mu$. Define $\T'$ as $(\EV', \EE',\paramState,\IV',\IE',\sigma')$
where,
%
\begin{align*}
\EV'		&= \hat{\fmap}(\V) \cap \EV \\
\IV'		&= \hat{\fmap}(\V) \cap \IV \\
\IE' 		&= \bigcup_{\langle u,f,v \rangle \in \IE} \fmap(u) \times \{ f \} \times \fmap(v) \\ 
\EE' 		&= \bigcup_{\langle u,f,v \rangle \in \EE} \fmap(u) \times \{ f \} \times \{ v\} \\
\sigma' 	&= \lambda x. \; \hat{\fmap}(\sigma(x)) \\
\end{align*}
%
$\simplify(\T) = \removeEscNodes(\T')$.

The computationally expensive part in the $\simplify$ operation is the determination of $\mu$ which involves a lfp computation.
A naive implementation of the constraint~\ref{eq:alias-rule} would repeatedly scan through all pairs of internal
and external edges in $\T$, check if the antecedent holds and apply the consequent 
as long as the mapping $\mu$ changes. 
In the following, the constraint~\ref{eq:alias-rule} is re-formulated in a different form so that 
its computation can be incrementalized.
In principle, the re-formulation is obtained by eliminating $\mu$ from the constraint~\ref{eq:alias-rule}
(we nevertheless compute $\mu$ for reasons explained later).

Say the computation of lfp of $\mu$ happens iteratively starting from $\emptyset$.
Let the value of $\mu$ in the iteration $i$ be $\mu_i$ (where, $\mu_0 = \emptyset$).
The lfp of $\mu$ is the value of $\mu_i$ such that $\mu_i = \mu_{i-1}$.
$\mu_i$ is defined using $\mu_{i-1}$ as follows:
%
\begin{align}
\mu_0(x) &= \emptyset \\
x \in \V & \Rightarrow  x \in \mu_1(x) \\
\nonumber
\langle u,f,v \rangle \in \IE, \langle u',f,v' \rangle \in \EE, \\
\mu_{i-1}(u) \cap \mu_{i-1}(u') \ne \emptyset  & \Rightarrow \mu_{i-1}(v) \subseteq \mu_i(v') 
\end{align}

Now, let us combine the computation of $\IE'$ and $\EE'$  with that of $\mu_i$  as follows:
(For now ignore $\sigma$ and consider only the edges)
%
\begin{align}
\mu_0(x) &= \emptyset \\
\IE 	 & \subseteq \IE' \\
\EE 	 & \subseteq \EE' \\
x \in \V & \Rightarrow  x \in \mu_1(x) \\
\nonumber
\langle u,f,v \rangle \in \IE, \langle u',f,v' \rangle \in \EE, \\
\nonumber
\mu_{i-1}(u) \cap \mu_{i-1}(u') \ne \emptyset  & \Rightarrow \mu_{i-1}(v) \subseteq \mu_i(v'),  \\
\nonumber
& \bigcup_{\langle x,f,v' \rangle \in \IE} \mu_{i-1}(x) \times \{ f \} \times \mu_{i-1}(v) \subseteq \IE', \\
\nonumber
& \bigcup_{\langle v',f,x \rangle \in \IE} \mu_{i-1}(v) \times \{ f \} \times \mu_{i-1}(x) \subseteq \IE', \\
& \bigcup_{\langle v',f,x \rangle \in \EE} \mu_{i-1}(v) \times \{ f \} \times \{  x \} \subseteq \EE' 
\label{eq:new-alias}
\end{align}
%
We define $\EV'$,$\IV'$ and $\sigma'$ as before using the lfp of $\mu$.
In the above constraints, whenever a node $u$ is added to $\mu_i(v)$ (for any $v$), 
all the internal and external edges on $v$ are  translated to $u$, eagerly.

The following invariants hold in the above constraints.
%
\begin{align}
\nonumber
\langle u,f,v \rangle \in \IE, \langle u',f,v' \rangle \in \EE,\mu_{i}(u) \cap \mu_{i}(u') \ne \emptyset \\
\qquad \implies \exists r. \langle r,f,v \rangle \in \IE', \langle r,f,v' \rangle \in \EE' \label{eq:inv1}
\end{align}
%
For any iteration $i$,
\begin{align}
\langle u,f,v \rangle \in \IE' & \implies \mu_{i-1}(u) \times \{ f \} \times \mu_{i-1}(v) \subseteq \IE' \label{eq:inv2}
\end{align}
%
Using invariant~\ref{eq:inv1}, the constraint~\ref{eq:new-alias} can be rewritten as follows:
%
\begin{align}
\nonumber
\langle r,f,v \rangle \in \IE', \langle r,f,v' \rangle \in \EE' 
						 & \Rightarrow \mu_{i-1}(v) \subseteq \mu_i(v'),  \\
\nonumber
& \bigcup_{\langle x,f,v' \rangle \in \IE} \mu_{i-1}(x) \times \{ f \} \times \mu_{i-1}(v) \subseteq \IE', \\
\nonumber
& \bigcup_{\langle v',f,x \rangle \in \IE} \mu_{i-1}(v) \times \{ f \} \times \mu_{i-1}(x) \subseteq \IE', \\
& \bigcup_{\langle v',f,x \rangle \in \EE} \mu_{i-1}(v) \times \{ f \} \times \{  x \} \subseteq \EE' 
\end{align}
%
By invariant~\ref{eq:inv2}, $\langle r,f,v \rangle \in \IE'$ implies that 
$\{ r \} \times \{ f \} \times \mu_{i-1}(v) \subseteq \IE'$ and 
$\mu_{i-1}(r) \times \{ f \} \times \{ v \} \subseteq \IE'$.
Hence, the above constraint can be rewritten as follows.
%
\begin{align}
\nonumber
 \{ r \} \times \{ f \} \times A \subseteq \IE', \langle r,f,v' \rangle \in \EE' 
						 & \Rightarrow A \subseteq \mu_i(v'),  \\
\nonumber
& \bigcup_{\langle x,f,v' \rangle \in \IE'} \{ x \} \times \{ f \} \times A \subseteq \IE', \\
\nonumber
& \bigcup_{\langle v',f,x \rangle \in \IE'} A \times \{ f \} \times \{ x \} \subseteq \IE', \\
& \bigcup_{\langle v',f,x \rangle \in \EE} A \times \{ f \} \times \{  x \} \subseteq \EE' 
\end{align}
%
Again, by invariant~\ref{eq:inv2}, the above can be further optimized as follows:
%
\begin{align}
\nonumber
& \{ r \} \times \{ f \} \times A \subseteq \IE', \langle r,f,v' \rangle \in \EE' \Rightarrow \\
\nonumber
& \qquad \text{let } B = A \setminus \mu_{i-1}(v'), \\
\nonumber
& \qquad B \subseteq \mu_i(v'),  \\
\nonumber
& \qquad \bigcup_{\langle x,f,v' \rangle \in \IE'} \{ x \} \times \{ f \} \times B \subseteq \IE', \\
\nonumber
& \qquad \bigcup_{\langle v',f,x \rangle \in \IE'} B \times \{ f \} \times \{ x \} \subseteq \IE', \\
& \qquad \bigcup_{\langle v',f,x \rangle \in \EE'} B \times \{ f \} \times \{  x \} \subseteq \EE' \label{eq:alias-final}
\end{align}

Notice that in the constraint~\ref{eq:alias-final}, computing $\mu$ is not strictly necessary (still it helps
in optimizing the computation). However, we compute $\mu$ in order to compute the other components 
of the transformer graphs (\eg{} $\sigma'$, vertex sets, may write sets for purity analysis etc.) using the 
earlier equations.

Constraint~\ref{eq:alias-final} can be incrementally evaluated as described below. 
Initially, we consider all vertices in the initial graph $\T$ and check if the antecedent applies. 
If yes, we perform the consequent. 
Whenever, a new internal edge $\langle r,f,v \rangle$ is added, we consider every external edge
$\langle r,f,v' \rangle$ from $r$. Clearly, the internal and external edges satisfy the antecedent.
Also, they cannot have been processed earlier since $\langle r,f,v \rangle$ is a newly added edge.
Hence, we perform the consequent for the internal /external edge pair.
Similarly, whenever a new external edge $\langle r,f,v \rangle$ is added, 
we consider all the internal edges on $r$ and trigger the consequent. 
In a nutshell, the incremental algorithm keeps track of the newly added internal and external edges 
and, in every iteration, identifies the new relations that satisfy the antecedent and triggers the 
consequent only for those relations.
In contrast, the direct implementation of the original constraints (\ref{eq:alias-rule}) 
has to scan all pairs of internal and external edges whenever there is a change to $\mu$. 

\end{document}